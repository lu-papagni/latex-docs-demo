\documentclass{article}

\usepackage[utf8]{inputenc}
\usepackage{geometry}
\usepackage{graphicx}
\usepackage{listings}
\usepackage{xcolor}
\usepackage{tikz}
\usepackage{hyperref}
\usepackage{cite}

\usetikzlibrary{shapes.geometric, arrows}
\tikzstyle{startstop} = [rectangle, rounded corners, minimum width=3cm, minimum height=1cm,text centered, draw=black, fill=red!30]

% Code highlighting
\lstset{
    language=Python,
    basicstyle=\ttfamily\small,
    keywordstyle=\color{blue},
    commentstyle=\color{green!60!black},
    stringstyle=\color{red},
    numbers=left,
    numberstyle=\tiny,
    frame=single,
    breaklines=true,
    captionpos=b
}

\title{FizzBuzz Algorithm Documentation}
\author{Luca}
\date{\today}



\begin{document}

\maketitle

\tableofcontents
\newpage

\section{Introduction}
\label{sec:intro}

The FizzBuzz algorithm is a classic programming exercise used to test basic understanding of loops, conditionals, and modular arithmetic. This document provides a comprehensive implementation and explanation of the FizzBuzz algorithm in Python.

The algorithm iterates through numbers from 1 to 100 and applies the following rules:
\begin{itemize}
    \item If the number is divisible by 3, output "Fizz"
    \item If the number is divisible by 5, output "Buzz"
    \item If the number is divisible by both 3 and 5, output "FizzBuzz"
    \item Otherwise, output the number itself
\end{itemize}

\section{Algorithm Overview}
\label{sec:overview}

The core logic can be visualized in the flowchart shown in Figure~\ref{fig:flowchart}.

\begin{figure}[h]
\centering
\begin{tikzpicture}[node distance=2cm]
    \node (start) [startstop] {Start};
    \node (input) [io, below of=start] {Input number n};
    \node (cond1) [decision, below of=input, yshift=-0.5cm] {n \% 3 == 0?};
    \node (cond2) [decision, right of=cond1, xshift=2cm] {n \% 5 == 0?};
    \node (fizzbuzz) [process, below of=cond2] {Output "FizzBuzz"};
    \node (fizz) [process, left of=fizzbuzz, xshift=-2cm] {Output "Fizz"};
    \node (buzz) [process, right of=fizzbuzz, xshift=2cm] {Output "Buzz"};
    \node (number) [process, below of=cond1] {Output n};
    \node (end) [startstop, below of=number] {End};

    \draw [arrow] (start) -- (input);
    \draw [arrow] (input) -- (cond1);
    \draw [arrow] (cond1) -- node[anchor=east] {Yes} (cond2);
    \draw [arrow] (cond2) -- node[anchor=south] {Yes} (fizzbuzz);
    \draw [arrow] (cond2) -- node[anchor=north] {No} (fizz);
    \draw [arrow] (cond1) -- node[anchor=east] {No} (number);
    \draw [arrow] (fizzbuzz) -- (end);
    \draw [arrow] (fizz) -- (end);
    \draw [arrow] (buzz) -- (end);
    \draw [arrow] (number) -- (end);
\end{tikzpicture}
\caption{FizzBuzz Algorithm Flowchart}
\label{fig:flowchart}
\end{figure}

\section{Code Implementation}
\label{sec:code}

The implementation consists of two main functions: \texttt{fizzbuzz()} and \texttt{main()}.

\subsection{The fizzbuzz Function}

The \texttt{fizzbuzz} function takes an integer as input and returns the appropriate string based on the divisibility rules.

\lstinputlisting[caption=FizzBuzz Function, firstline=13, lastline=30]{../fizzbuzz.py}

\subsection{The main Function}

The \texttt{main} function iterates through numbers 1 to 100 and prints the result of calling \texttt{fizzbuzz} on each.

\lstinputlisting[caption=Main Function, firstline=32, lastline=38]{../fizzbuzz.py}

\section{Sample Output}
\label{sec:sample}

Table~\ref{tab:sample} shows the output for the first 20 numbers.

\begin{table}[h]
\centering
\caption{Sample FizzBuzz Output}
\label{tab:sample}
\begin{tabular}{|c|c|}
\hline
Number & Output \\
\hline
1 & 1 \\
2 & 2 \\
3 & Fizz \\
4 & 4 \\
5 & Buzz \\
6 & Fizz \\
7 & 7 \\
8 & 8 \\
9 & Fizz \\
10 & Buzz \\
11 & 11 \\
12 & Fizz \\
13 & 13 \\
14 & 14 \\
15 & FizzBuzz \\
16 & 16 \\
17 & 17 \\
18 & Fizz \\
19 & 19 \\
20 & Buzz \\
\hline
\end{tabular}
\end{table}

\section{Conclusion}
\label{sec:conclusion}

This implementation demonstrates a clean, modular approach to the FizzBuzz problem. The code is well-documented and follows Python best practices.

For more information on similar algorithms, see~\cite{knuth1997art}.

\bibliography{references}
\bibliographystyle{plain}

\end{document}
